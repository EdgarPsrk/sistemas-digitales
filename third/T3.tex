\documentclass[a4paper, 12pt]{article}

\usepackage[T1]{fontenc}
\usepackage[utf8]{inputenc}
\usepackage[spanish, mexico]{babel}
\usepackage[style=mexican]{csquotes}
\usepackage[margin=2cm,top=2cm,includefoot]{geometry}
\usepackage[spanish, ruled, linesnumbered, lined]{algorithm2e}
\usepackage{amsmath, amsfonts, amssymb, amsthm, amsbsy, cancel}
\usepackage{microtype, parskip}
\usepackage{float, graphicx, subcaption}
\usepackage{circuitikz, tikz, pgfplots}
\usepackage{xcolor}
\usepackage{array, booktabs, multicol, multirow, tabularx}
\usepackage{hyperref, url}
\usepackage{siunitx}
\usepackage{tcolorbox}
\usepackage[style=ieee]{biblatex}
%definir el estilo de lapagina
\usepackage{fancyhdr}
%code
\usepackage{listings}


%variables de color
% \definecolor{greenPortada}{HTML}{69A84F}
\definecolor{lineCabecera}{HTML}{5DADE2}

%cabecera
\setlength{\headheight}{40pt}
\pagestyle{fancy}
\fancyhf{}
\renewcommand{\headrulewidth}{3pt}
%\renewcommand{\headrule}{\color{lineCabecera}\hrulefill}
\renewcommand{\headrule}{\hbox to \headwidth{\color{lineCabecera}\leaders\hrule height \headrulewidth\hfill}}

%variables globales
\newcommand{\esquema}{img/esq.jpg}
\newcommand{\var}{img/jflap.png}
\newcommand{\imgA}{img/dt1.png}
\newcommand{\imgB}{img/dt2.png}
\newcommand{\codeA}{code/clock_pulse.vhd}
\newcommand{\codeB}{code/debounce.vhd}
\newcommand{\codeC}{code/doorlock.vhd}
\newcommand{\codeD}{code/frequency_divider.vhd}
\newcommand{\codeE}{code/system.vhd}



%gestion de hipervinculos
\hypersetup{
    breaklinks=true,
    colorlinks=true,
    citecolor=black,
    filecolor=magenta,
    linkcolor=black, 
    urlcolor=cyan
}
%gestor de codigo
\lstset{
    % language=VHDL,
    % breaklines=true, % Activa el ajuste de líneas
    % numbers=left, % Números de línea a la izquierda
    % frame=single, % Borde alrededor del código
    language=VHDL,
    basicstyle=\ttfamily,
    keywordstyle=\color{blue},    % Color de las palabras clave
    commentstyle=\color{green},   % Color de los comentarios
    numberstyle=\tiny\color{gray},% Color de los números de línea
    stringstyle=\color{red},      % Color de las cadenas de texto
    breaklines=true,
    numbers=left,
    frame=single,
}

\title{Sistema en VHDL con suma, resta y comparación de 2 palabras a 4 bits}
\author{Universidad Nacional Autónoma de México.\\Facultad de Estudios Superiores Cuatitlán.\\Palomino Alfonso Edgar.}
\date{\today}

\cfoot{\thepage}

\begin{document}
    \maketitle  
    \vspace{2ex}

    \section{Introducción.}
    En el campo de la ingeniería y la informática, las operaciones aritméticas fundamentales de suma, resta y comparación desempeñan un papel esencial en una amplia gama de aplicaciones. En este proyecto, presentaremos el diseño y la implementación de un sistema digital en VHDL que realiza estas operaciones con palabras de 4 bits. Tanto el sumador como el restador se basan en un diseño tradicional de acarreo completo, mientras que el comparador utiliza un enfoque de comparación de magnitud de 4 bits.A comparación de la simulación en este sistema tiene la capacidad para manejar números en su totalidad, incluyendo valores negativos. A diferencia del diseños pasado que restringen las operaciones solo a números positivos, nuestro sistema en VHDL permite operar con números positivos y negativos, también el comparador tiene la tarea de determinar si dos números son iguales, si uno es mayor que el otro o si uno es menor que el otro. A lo largo de este informe, profundizaremos en el diseño y el funcionamiento de estas operaciones cruciales en el contexto de sistemas digitales basados en VHDL.
    
    \section{Descripción del método.}
    La solución propuesta para el sistema consiste en una entrada binaria a un sumador completo de 4 bits, el cual produce una salida en formato \emph{BCD}. Esto es necesario ya que se requiere pasarlo a un formato BCD para mostrarlo en tres displays de 7 segmentos. Se condiciona la salida de este sumador de tal manera que si es menor que nueve, se le agregan ceros para tener 8 bits en la salida (4 bits para cada display de 7 segmentos). En cualquier otro caso, se verifica que la salida no sea mayor que nueve para asegurar una salida \emph{BCD válida}. Si no se cumple esta condición, se le suman 6 en binario hasta obtener un BCD válido. Una vez que esto se cumple, se pueden segmentar las combinaciones necesarias en 4 bits para su representación en los displays de 7 segmentos.
    La resta presenta el problema de que si se requiere representar números negativos de 2 dígitos decimales, se necesitaría un display adicional para representar el signo, lo cual se contempla en este sistema. Entonces, si $x > y$, se realiza la resta y se representa de la misma manera que en la suma, en caso contrario se realiza el complemento a 2 para obtener el valor correspondiente a la resta verificando que el resultado sea menor a 10, en este caso el tercer display mostrará el signo para el resultado. 
    Para el comparador, se necesita un cambio ligero en la lógica, ya que debe mostrar \emph{'G'} si $x > y$, \emph{'E'} si $x = y$ y \emph{'L'} $si x < y$. Por lo tanto, para cada caso se debe proporcionar una representación diferente a la suma y resta. Esto implica que en el decodificador de 7 segmentos se deben restringir tres combinaciones para realizar esta representación. Esto se abordará dejando las letras necesarias utilizando diferentes codificadores.
    La elección de la operación necesaria se realiza mediante un multiplexor. Si configuramos el multiplexor en 01, se realizará la suma de las entradas; si configuramos en 10, obtendremos la resta. Por último, el bit más significativo es vital para la comparación. Si configuramos el multiplexor en $x1$, se generará la comparación, teniendo como resultado la figura~\ref{fig:top-level}.

    \begin{figure}[H]
        \centering
        \includegraphics[width=\textwidth,keepaspectratio]{\esquema}
        \caption{Esquematico del sistema}
        \label{fig:top-level}
    \end{figure}

    \section{Experimentos}
    Las salidas del diseño se dejan en el apéndice A, donde se pueden analizar las variables que se muestran del archivo top en la figura~\ref{fig:variables}, donde se aprecian distintas señales pero solo no interesan las correspondientes a los display, ya que estos valores son las representaciones de la salida, o los valores que deberíamos tener, debido a que el selector no está dentro de ninguno de los casos donde se mande la salida de ninguna de las operaciones podemos parecia en la figura~\ref{fig:signal}. que todos los segmentos están en alto, es decir apagados, además podemos apreciar cuales son las valores de las 2 entradas correspondientes que son $a = 10$ y $b = 5$, si generamos la conversión de los valores de los display, correspondientes a 12 para \emph{display0} y 79 para el \emph{display1} podemos verificar que tenemos 1 y 5 dando como resultado la suma de las dos entradas en codigo BCD, ademas podemos notar que el \emph{display2} está apagado, de esta manera puedes verificar que la salida del sistema es correcta, podemos jugar con el archivo test bench para realizar toda las pruebas con las combinaciones generando las salidas correctas en los todos los display.

    \section{Conclusión}
    El proceso de crear un diseño digital utilizando VHDL implica un aprendizaje significativo. Se requiere comprender la sintaxis de VHDL, los conceptos de diseño digital y cómo implementarlos en un entorno como Vivado, creando un sistema que realiza múltiples operaciones (suma, resta y comparación), destacando la importancia del diseño modular en VHDL. 
    Poder dividir el sistema en módulos o entidades más pequeñas facilita el diseño, la depuración y la reutilización de código y debido a que no tenemos acceso a la tarjeta la simulación es una parte esencial del proceso de diseño en VHDL, aunque también se debe tomar en cuenta que antes de implementar el diseño en hardware, es crucial realizar una simulación exhaustiva para verificar su funcionalidad y detectar posibles errores, además la implementación en hardware a través de herramientas como Vivado implica la síntesis y optimización del diseño. Esto se traduce en la generación de un circuito físico a partir del código VHDL con la optimización de recursos como área y velocidad, por último la transición de un diseño físico a un diseño digital implica un cambio de paradigma importante, que nos permite una mayor flexibilidad y la posibilidad de realizar cambios más rápidamente en el diseño sin modificar hardware físico.    

    \appendix
    \section*{Apéndice A: Imagenes de la simulación}
    \begin{figure}[H]
        \centering
        \includegraphics[width=\textwidth,height=12cm,keepaspectratio]{\var}
        \caption{Variables y señales del archivo top}
        \label{fig:variables}
    \end{figure}

    \begin{figure}[H]
        \centering
        \includegraphics[width=\textwidth]{\imgB}
        \caption{Señales de salida}
        \label{fig:signal}
    \end{figure} 

    \section*{Apéndice B: Codigo VHDL}
    Código del módulo con pulsos de reloj verificados
    \lstinputlisting[language=VHDL]{\codeA}

    Código del módulo eliminador de la señal de rebote de los botones
    \lstinputlisting[language=VHDL]{\codeB}

    Código del módulo de la maquina de estados
    \lstinputlisting[language=VHDL]{\codeC}

    Código del módulo preescaler
    \lstinputlisting[language=VHDL]{\codeD}

    Código del módulo top conjuntador de módulos
    \lstinputlisting[language=VHDL]{\codeE}

    
\end{document}
